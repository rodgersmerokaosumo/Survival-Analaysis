% Options for packages loaded elsewhere
\PassOptionsToPackage{unicode}{hyperref}
\PassOptionsToPackage{hyphens}{url}
%
\documentclass[
]{article}
\usepackage{amsmath,amssymb}
\usepackage{lmodern}
\usepackage{iftex}
\ifPDFTeX
  \usepackage[T1]{fontenc}
  \usepackage[utf8]{inputenc}
  \usepackage{textcomp} % provide euro and other symbols
\else % if luatex or xetex
  \usepackage{unicode-math}
  \defaultfontfeatures{Scale=MatchLowercase}
  \defaultfontfeatures[\rmfamily]{Ligatures=TeX,Scale=1}
\fi
% Use upquote if available, for straight quotes in verbatim environments
\IfFileExists{upquote.sty}{\usepackage{upquote}}{}
\IfFileExists{microtype.sty}{% use microtype if available
  \usepackage[]{microtype}
  \UseMicrotypeSet[protrusion]{basicmath} % disable protrusion for tt fonts
}{}
\makeatletter
\@ifundefined{KOMAClassName}{% if non-KOMA class
  \IfFileExists{parskip.sty}{%
    \usepackage{parskip}
  }{% else
    \setlength{\parindent}{0pt}
    \setlength{\parskip}{6pt plus 2pt minus 1pt}}
}{% if KOMA class
  \KOMAoptions{parskip=half}}
\makeatother
\usepackage{xcolor}
\usepackage[margin=1in]{geometry}
\usepackage{color}
\usepackage{fancyvrb}
\newcommand{\VerbBar}{|}
\newcommand{\VERB}{\Verb[commandchars=\\\{\}]}
\DefineVerbatimEnvironment{Highlighting}{Verbatim}{commandchars=\\\{\}}
% Add ',fontsize=\small' for more characters per line
\usepackage{framed}
\definecolor{shadecolor}{RGB}{248,248,248}
\newenvironment{Shaded}{\begin{snugshade}}{\end{snugshade}}
\newcommand{\AlertTok}[1]{\textcolor[rgb]{0.94,0.16,0.16}{#1}}
\newcommand{\AnnotationTok}[1]{\textcolor[rgb]{0.56,0.35,0.01}{\textbf{\textit{#1}}}}
\newcommand{\AttributeTok}[1]{\textcolor[rgb]{0.77,0.63,0.00}{#1}}
\newcommand{\BaseNTok}[1]{\textcolor[rgb]{0.00,0.00,0.81}{#1}}
\newcommand{\BuiltInTok}[1]{#1}
\newcommand{\CharTok}[1]{\textcolor[rgb]{0.31,0.60,0.02}{#1}}
\newcommand{\CommentTok}[1]{\textcolor[rgb]{0.56,0.35,0.01}{\textit{#1}}}
\newcommand{\CommentVarTok}[1]{\textcolor[rgb]{0.56,0.35,0.01}{\textbf{\textit{#1}}}}
\newcommand{\ConstantTok}[1]{\textcolor[rgb]{0.00,0.00,0.00}{#1}}
\newcommand{\ControlFlowTok}[1]{\textcolor[rgb]{0.13,0.29,0.53}{\textbf{#1}}}
\newcommand{\DataTypeTok}[1]{\textcolor[rgb]{0.13,0.29,0.53}{#1}}
\newcommand{\DecValTok}[1]{\textcolor[rgb]{0.00,0.00,0.81}{#1}}
\newcommand{\DocumentationTok}[1]{\textcolor[rgb]{0.56,0.35,0.01}{\textbf{\textit{#1}}}}
\newcommand{\ErrorTok}[1]{\textcolor[rgb]{0.64,0.00,0.00}{\textbf{#1}}}
\newcommand{\ExtensionTok}[1]{#1}
\newcommand{\FloatTok}[1]{\textcolor[rgb]{0.00,0.00,0.81}{#1}}
\newcommand{\FunctionTok}[1]{\textcolor[rgb]{0.00,0.00,0.00}{#1}}
\newcommand{\ImportTok}[1]{#1}
\newcommand{\InformationTok}[1]{\textcolor[rgb]{0.56,0.35,0.01}{\textbf{\textit{#1}}}}
\newcommand{\KeywordTok}[1]{\textcolor[rgb]{0.13,0.29,0.53}{\textbf{#1}}}
\newcommand{\NormalTok}[1]{#1}
\newcommand{\OperatorTok}[1]{\textcolor[rgb]{0.81,0.36,0.00}{\textbf{#1}}}
\newcommand{\OtherTok}[1]{\textcolor[rgb]{0.56,0.35,0.01}{#1}}
\newcommand{\PreprocessorTok}[1]{\textcolor[rgb]{0.56,0.35,0.01}{\textit{#1}}}
\newcommand{\RegionMarkerTok}[1]{#1}
\newcommand{\SpecialCharTok}[1]{\textcolor[rgb]{0.00,0.00,0.00}{#1}}
\newcommand{\SpecialStringTok}[1]{\textcolor[rgb]{0.31,0.60,0.02}{#1}}
\newcommand{\StringTok}[1]{\textcolor[rgb]{0.31,0.60,0.02}{#1}}
\newcommand{\VariableTok}[1]{\textcolor[rgb]{0.00,0.00,0.00}{#1}}
\newcommand{\VerbatimStringTok}[1]{\textcolor[rgb]{0.31,0.60,0.02}{#1}}
\newcommand{\WarningTok}[1]{\textcolor[rgb]{0.56,0.35,0.01}{\textbf{\textit{#1}}}}
\usepackage{longtable,booktabs,array}
\usepackage{calc} % for calculating minipage widths
% Correct order of tables after \paragraph or \subparagraph
\usepackage{etoolbox}
\makeatletter
\patchcmd\longtable{\par}{\if@noskipsec\mbox{}\fi\par}{}{}
\makeatother
% Allow footnotes in longtable head/foot
\IfFileExists{footnotehyper.sty}{\usepackage{footnotehyper}}{\usepackage{footnote}}
\makesavenoteenv{longtable}
\usepackage{graphicx}
\makeatletter
\def\maxwidth{\ifdim\Gin@nat@width>\linewidth\linewidth\else\Gin@nat@width\fi}
\def\maxheight{\ifdim\Gin@nat@height>\textheight\textheight\else\Gin@nat@height\fi}
\makeatother
% Scale images if necessary, so that they will not overflow the page
% margins by default, and it is still possible to overwrite the defaults
% using explicit options in \includegraphics[width, height, ...]{}
\setkeys{Gin}{width=\maxwidth,height=\maxheight,keepaspectratio}
% Set default figure placement to htbp
\makeatletter
\def\fps@figure{htbp}
\makeatother
\setlength{\emergencystretch}{3em} % prevent overfull lines
\providecommand{\tightlist}{%
  \setlength{\itemsep}{0pt}\setlength{\parskip}{0pt}}
\setcounter{secnumdepth}{-\maxdimen} % remove section numbering
\ifLuaTeX
  \usepackage{selnolig}  % disable illegal ligatures
\fi
\IfFileExists{bookmark.sty}{\usepackage{bookmark}}{\usepackage{hyperref}}
\IfFileExists{xurl.sty}{\usepackage{xurl}}{} % add URL line breaks if available
\urlstyle{same} % disable monospaced font for URLs
\hypersetup{
  pdftitle={Analysis},
  hidelinks,
  pdfcreator={LaTeX via pandoc}}

\title{Analysis}
\author{}
\date{\vspace{-2.5em}2023-05-29}

\begin{document}
\maketitle

\hypertarget{data-import}{%
\subsubsection{Data Import}\label{data-import}}

\begin{Shaded}
\begin{Highlighting}[]
\CommentTok{\# Specify the file path}
\NormalTok{file\_path }\OtherTok{\textless{}{-}} \StringTok{"echocardiogram.data"}

\CommentTok{\# Read the file into a data frame}
\NormalTok{data }\OtherTok{\textless{}{-}} \FunctionTok{read.csv}\NormalTok{(file\_path, }\AttributeTok{header =} \ConstantTok{FALSE}\NormalTok{, }\AttributeTok{na.strings =} \StringTok{"?"}\NormalTok{)}

\CommentTok{\# Assuming \textasciigrave{}data\textasciigrave{} is your data frame}
\NormalTok{column\_names }\OtherTok{\textless{}{-}} \FunctionTok{c}\NormalTok{(}\StringTok{"survival"}\NormalTok{, }\StringTok{"still{-}alive"}\NormalTok{, }\StringTok{"age{-}at{-}heart{-}attack"}\NormalTok{, }\StringTok{"pericardial{-}effusion"}\NormalTok{, }\StringTok{"fractional{-}shortening"}\NormalTok{, }\StringTok{"epss"}\NormalTok{, }\StringTok{"lvdd"}\NormalTok{, }\StringTok{"wall{-}motion{-}score"}\NormalTok{, }\StringTok{"wall{-}motion{-}index"}\NormalTok{, }\StringTok{"mult"}\NormalTok{, }\StringTok{"name"}\NormalTok{, }\StringTok{"group"}\NormalTok{, }\StringTok{"alive{-}at{-}1"}\NormalTok{)}

\FunctionTok{colnames}\NormalTok{(data) }\OtherTok{\textless{}{-}}\NormalTok{ column\_names}
\FunctionTok{head}\NormalTok{(data)}
\end{Highlighting}
\end{Shaded}

\begin{verbatim}
##   survival still-alive age-at-heart-attack pericardial-effusion
## 1       11           0                  71                    0
## 2       19           0                  72                    0
## 3       16           0                  55                    0
## 4       57           0                  60                    0
## 5       19           1                  57                    0
## 6       26           0                  68                    0
##   fractional-shortening   epss  lvdd wall-motion-score wall-motion-index  mult
## 1                 0.260  9.000 4.600                14              1.00 1.000
## 2                 0.380  6.000 4.100                14              1.70 0.588
## 3                 0.260  4.000 3.420                14              1.00 1.000
## 4                 0.253 12.062 4.603                16              1.45 0.788
## 5                 0.160 22.000 5.750                18              2.25 0.571
## 6                 0.260  5.000 4.310                12              1.00 0.857
##   name group alive-at-1
## 1 name     1          0
## 2 name     1          0
## 3 name     1          0
## 4 name     1          0
## 5 name     1          0
## 6 name     1          0
\end{verbatim}

In survival analysis, the primary objective is to estimate the survival
distribution and analyze the impact of various variables on the time it
takes for an event to occur. This type of analysis is commonly used in
medical research, epidemiology, and other fields where understanding
time-to-event data is crucial. Parameter estimation in survival analysis
involves estimating the parameters of the chosen survival distribution,
such as the hazard function or survival function, using statistical
methods like maximum likelihood estimation.

On the other hand, the Poisson distribution is a probability
distribution that models the number of events occurring in a fixed
interval of time or space. It is commonly used when dealing with count
data, such as the number of occurrences of a specific event. The Poisson
distribution estimates the rate of event occurrence based on the average
number of events in the given interval. Parameter estimation in the
Poisson distribution involves estimating the rate parameter, which
represents the average event rate.

While both survival analysis and the Poisson distribution deal with
event occurrence, they differ in their approach and focus. Survival
analysis focuses on modeling the time until an event occurs and
understanding the factors influencing it, whereas the Poisson
distribution focuses on estimating the rate of event occurrence in a
fixed interval.

\begin{Shaded}
\begin{Highlighting}[]
\CommentTok{\# Explore the structure of the dataset}
\NormalTok{data }\OtherTok{=}\NormalTok{ data }\SpecialCharTok{|\textgreater{}} \FunctionTok{clean\_names}\NormalTok{()}
\NormalTok{data }\OtherTok{\textless{}{-}}\NormalTok{ data }\SpecialCharTok{\%\textgreater{}\%}
  \FunctionTok{mutate\_all}\NormalTok{(}\SpecialCharTok{\textasciitilde{}}\FunctionTok{ifelse}\NormalTok{(. }\SpecialCharTok{==} \StringTok{"?"}\NormalTok{, }\ConstantTok{NA}\NormalTok{, .))}

\NormalTok{data }\OtherTok{\textless{}{-}}\NormalTok{ data }\SpecialCharTok{\%\textgreater{}\%}
  \FunctionTok{select}\NormalTok{(}\SpecialCharTok{{-}}\NormalTok{name)}
\NormalTok{data }\SpecialCharTok{|\textgreater{}} \FunctionTok{glimpse}\NormalTok{()}
\end{Highlighting}
\end{Shaded}

\begin{verbatim}
## Rows: 133
## Columns: 12
## $ survival              <dbl> 11.00, 19.00, 16.00, 57.00, 19.00, 26.00, 13.00,~
## $ still_alive           <int> 0, 0, 0, 0, 1, 0, 0, 0, 0, 0, 1, 0, 0, 0, 1, 0, ~
## $ age_at_heart_attack   <dbl> 71.000, 72.000, 55.000, 60.000, 57.000, 68.000, ~
## $ pericardial_effusion  <int> 0, 0, 0, 0, 0, 0, 0, 0, 0, 0, 0, 1, 0, 0, 0, 1, ~
## $ fractional_shortening <dbl> 0.260, 0.380, 0.260, 0.253, 0.160, 0.260, 0.230,~
## $ epss                  <dbl> 9.000, 6.000, 4.000, 12.062, 22.000, 5.000, 31.0~
## $ lvdd                  <dbl> 4.600, 4.100, 3.420, 4.603, 5.750, 4.310, 5.430,~
## $ wall_motion_score     <dbl> 14.00, 14.00, 14.00, 16.00, 18.00, 12.00, 22.50,~
## $ wall_motion_index     <dbl> 1.000, 1.700, 1.000, 1.450, 2.250, 1.000, 1.875,~
## $ mult                  <dbl> 1.000, 0.588, 1.000, 0.788, 0.571, 0.857, 0.857,~
## $ group                 <chr> "1", "1", "1", "1", "1", "1", "1", "1", "1", "1"~
## $ alive_at_1            <int> 0, 0, 0, 0, 0, 0, 0, 0, 0, 0, 1, 0, 0, 0, 1, 0, ~
\end{verbatim}

\begin{Shaded}
\begin{Highlighting}[]
\NormalTok{data }\OtherTok{\textless{}{-}}\NormalTok{ data }\SpecialCharTok{\%\textgreater{}\%}
  \FunctionTok{mutate}\NormalTok{(}
    \AttributeTok{still\_alive =} \FunctionTok{factor}\NormalTok{(still\_alive),}
    \AttributeTok{pericardial\_effusion =} \FunctionTok{factor}\NormalTok{(pericardial\_effusion),}
    \AttributeTok{alive\_at\_1 =} \FunctionTok{factor}\NormalTok{(alive\_at\_1),}
    \AttributeTok{group =} \FunctionTok{factor}\NormalTok{(group)}
\NormalTok{  )}

\NormalTok{data }\OtherTok{\textless{}{-}}\NormalTok{ data }\SpecialCharTok{\%\textgreater{}\%}
  \FunctionTok{mutate}\NormalTok{(}
    \AttributeTok{survival =} \FunctionTok{as.numeric}\NormalTok{(survival),}
    \AttributeTok{age\_at\_heart\_attack =} \FunctionTok{as.numeric}\NormalTok{(age\_at\_heart\_attack),}
    \AttributeTok{fractional\_shortening =} \FunctionTok{as.numeric}\NormalTok{(fractional\_shortening),}
    \AttributeTok{epss =} \FunctionTok{as.numeric}\NormalTok{(epss),}
    \AttributeTok{lvdd =} \FunctionTok{as.numeric}\NormalTok{(lvdd),}
    \AttributeTok{wall\_motion\_score =} \FunctionTok{as.numeric}\NormalTok{(wall\_motion\_score),}
    \AttributeTok{wall\_motion\_index =} \FunctionTok{as.numeric}\NormalTok{(wall\_motion\_index),}
    \AttributeTok{mult =} \FunctionTok{as.numeric}\NormalTok{(mult)}
\NormalTok{  )}

\FunctionTok{head}\NormalTok{(data)}
\end{Highlighting}
\end{Shaded}

\begin{verbatim}
##   survival still_alive age_at_heart_attack pericardial_effusion
## 1       11           0                  71                    0
## 2       19           0                  72                    0
## 3       16           0                  55                    0
## 4       57           0                  60                    0
## 5       19           1                  57                    0
## 6       26           0                  68                    0
##   fractional_shortening   epss  lvdd wall_motion_score wall_motion_index  mult
## 1                 0.260  9.000 4.600                14              1.00 1.000
## 2                 0.380  6.000 4.100                14              1.70 0.588
## 3                 0.260  4.000 3.420                14              1.00 1.000
## 4                 0.253 12.062 4.603                16              1.45 0.788
## 5                 0.160 22.000 5.750                18              2.25 0.571
## 6                 0.260  5.000 4.310                12              1.00 0.857
##   group alive_at_1
## 1     1          0
## 2     1          0
## 3     1          0
## 4     1          0
## 5     1          0
## 6     1          0
\end{verbatim}

\begin{Shaded}
\begin{Highlighting}[]
\FunctionTok{summary}\NormalTok{(data)}
\end{Highlighting}
\end{Shaded}

\begin{verbatim}
##     survival      still_alive age_at_heart_attack pericardial_effusion
##  Min.   : 0.030   0   :88     Min.   :35.00       0   :107            
##  1st Qu.: 7.875   1   :43     1st Qu.:57.00       1   : 24            
##  Median :23.500   NA's: 2     Median :62.00       77  :  1            
##  Mean   :22.183               Mean   :62.81       NA's:  1            
##  3rd Qu.:33.000               3rd Qu.:67.75                           
##  Max.   :57.000               Max.   :86.00                           
##  NA's   :3                    NA's   :7                               
##  fractional_shortening      epss            lvdd       wall_motion_score
##  Min.   :0.0100        Min.   : 0.00   Min.   :2.320   Min.   : 2.00    
##  1st Qu.:0.1500        1st Qu.: 7.00   1st Qu.:4.230   1st Qu.:11.00    
##  Median :0.2050        Median :11.00   Median :4.650   Median :14.00    
##  Mean   :0.2167        Mean   :12.16   Mean   :4.763   Mean   :14.44    
##  3rd Qu.:0.2700        3rd Qu.:16.10   3rd Qu.:5.300   3rd Qu.:16.50    
##  Max.   :0.6100        Max.   :40.00   Max.   :6.780   Max.   :39.00    
##  NA's   :9             NA's   :16      NA's   :12      NA's   :5        
##  wall_motion_index      mult         group    alive_at_1
##  Min.   :1.000     Min.   :0.1400       : 1   0   :50   
##  1st Qu.:1.000     1st Qu.:0.7140   1   :24   1   :24   
##  Median :1.216     Median :0.7860   2   :85   2   : 1   
##  Mean   :1.378     Mean   :0.7862   name: 1   NA's:58   
##  3rd Qu.:1.508     3rd Qu.:0.8570   NA's:22             
##  Max.   :3.000     Max.   :2.0000                       
##  NA's   :3         NA's   :4
\end{verbatim}

\begin{Shaded}
\begin{Highlighting}[]
\CommentTok{\# Calculate the number of missing values in each column}
\FunctionTok{colSums}\NormalTok{(}\FunctionTok{is.na}\NormalTok{(data))}
\end{Highlighting}
\end{Shaded}

\begin{verbatim}
##              survival           still_alive   age_at_heart_attack 
##                     3                     2                     7 
##  pericardial_effusion fractional_shortening                  epss 
##                     1                     9                    16 
##                  lvdd     wall_motion_score     wall_motion_index 
##                    12                     5                     3 
##                  mult                 group            alive_at_1 
##                     4                    22                    58
\end{verbatim}

\begin{Shaded}
\begin{Highlighting}[]
\CommentTok{\#impute missing values with the median}
\NormalTok{data }\OtherTok{\textless{}{-}}\NormalTok{ data }\SpecialCharTok{\%\textgreater{}\%}
  \FunctionTok{mutate}\NormalTok{(}\FunctionTok{across}\NormalTok{(}\FunctionTok{where}\NormalTok{(is.numeric), }\SpecialCharTok{\textasciitilde{}}\FunctionTok{replace\_na}\NormalTok{(., }\FunctionTok{mean}\NormalTok{(.))))}
\end{Highlighting}
\end{Shaded}

\begin{Shaded}
\begin{Highlighting}[]
\FunctionTok{library}\NormalTok{(tidyverse)}
\FunctionTok{library}\NormalTok{(knitr)}

\CommentTok{\# Assuming \textasciigrave{}data\textasciigrave{} is your data frame}

\CommentTok{\# View the summary statistics of numeric columns}
\NormalTok{summary\_stats }\OtherTok{\textless{}{-}}\NormalTok{ data }\SpecialCharTok{\%\textgreater{}\%}
  \FunctionTok{select}\NormalTok{(}\FunctionTok{where}\NormalTok{(is.numeric)) }\SpecialCharTok{\%\textgreater{}\%}
  \FunctionTok{summary}\NormalTok{()}

\CommentTok{\# Print the summary statistics in a table using kable}
\FunctionTok{kable}\NormalTok{(summary\_stats)}
\end{Highlighting}
\end{Shaded}

\begin{longtable}[]{@{}
  >{\raggedright\arraybackslash}p{(\columnwidth - 16\tabcolsep) * \real{0.0216}}
  >{\raggedright\arraybackslash}p{(\columnwidth - 16\tabcolsep) * \real{0.1079}}
  >{\raggedright\arraybackslash}p{(\columnwidth - 16\tabcolsep) * \real{0.1439}}
  >{\raggedright\arraybackslash}p{(\columnwidth - 16\tabcolsep) * \real{0.1583}}
  >{\raggedright\arraybackslash}p{(\columnwidth - 16\tabcolsep) * \real{0.1007}}
  >{\raggedright\arraybackslash}p{(\columnwidth - 16\tabcolsep) * \real{0.1007}}
  >{\raggedright\arraybackslash}p{(\columnwidth - 16\tabcolsep) * \real{0.1295}}
  >{\raggedright\arraybackslash}p{(\columnwidth - 16\tabcolsep) * \real{0.1295}}
  >{\raggedright\arraybackslash}p{(\columnwidth - 16\tabcolsep) * \real{0.1079}}@{}}
\toprule()
\begin{minipage}[b]{\linewidth}\raggedright
\end{minipage} & \begin{minipage}[b]{\linewidth}\raggedright
survival
\end{minipage} & \begin{minipage}[b]{\linewidth}\raggedright
age\_at\_heart\_attack
\end{minipage} & \begin{minipage}[b]{\linewidth}\raggedright
fractional\_shortening
\end{minipage} & \begin{minipage}[b]{\linewidth}\raggedright
epss
\end{minipage} & \begin{minipage}[b]{\linewidth}\raggedright
lvdd
\end{minipage} & \begin{minipage}[b]{\linewidth}\raggedright
wall\_motion\_score
\end{minipage} & \begin{minipage}[b]{\linewidth}\raggedright
wall\_motion\_index
\end{minipage} & \begin{minipage}[b]{\linewidth}\raggedright
mult
\end{minipage} \\
\midrule()
\endhead
& Min. : 0.030 & Min. :35.00 & Min. :0.0100 & Min. : 0.00 & Min. :2.320
& Min. : 2.00 & Min. :1.000 & Min. :0.1400 \\
& 1st Qu.: 7.875 & 1st Qu.:57.00 & 1st Qu.:0.1500 & 1st Qu.: 7.00 & 1st
Qu.:4.230 & 1st Qu.:11.00 & 1st Qu.:1.000 & 1st Qu.:0.7140 \\
& Median :23.500 & Median :62.00 & Median :0.2050 & Median :11.00 &
Median :4.650 & Median :14.00 & Median :1.216 & Median :0.7860 \\
& Mean :22.183 & Mean :62.81 & Mean :0.2167 & Mean :12.16 & Mean :4.763
& Mean :14.44 & Mean :1.378 & Mean :0.7862 \\
& 3rd Qu.:33.000 & 3rd Qu.:67.75 & 3rd Qu.:0.2700 & 3rd Qu.:16.10 & 3rd
Qu.:5.300 & 3rd Qu.:16.50 & 3rd Qu.:1.508 & 3rd Qu.:0.8570 \\
& Max. :57.000 & Max. :86.00 & Max. :0.6100 & Max. :40.00 & Max. :6.780
& Max. :39.00 & Max. :3.000 & Max. :2.0000 \\
& NA's :3 & NA's :7 & NA's :9 & NA's :16 & NA's :12 & NA's :5 & NA's :3
& NA's :4 \\
\bottomrule()
\end{longtable}

\begin{Shaded}
\begin{Highlighting}[]
\CommentTok{\# Visualize the distribution of numeric variables}
\NormalTok{numeric\_vars }\OtherTok{\textless{}{-}} \FunctionTok{names}\NormalTok{(data)[}\FunctionTok{sapply}\NormalTok{(data, is.numeric)]}
\CommentTok{\# Create histograms for numeric variables}
\NormalTok{histograms }\OtherTok{\textless{}{-}}\NormalTok{ data }\SpecialCharTok{\%\textgreater{}\%}
  \FunctionTok{select}\NormalTok{(}\FunctionTok{all\_of}\NormalTok{(numeric\_vars)) }\SpecialCharTok{\%\textgreater{}\%}
  \FunctionTok{pivot\_longer}\NormalTok{(}\FunctionTok{everything}\NormalTok{(), }\AttributeTok{names\_to =} \StringTok{"Variable"}\NormalTok{, }\AttributeTok{values\_to =} \StringTok{"Value"}\NormalTok{) }\SpecialCharTok{\%\textgreater{}\%}
  \FunctionTok{ggplot}\NormalTok{(}\FunctionTok{aes}\NormalTok{(}\AttributeTok{x =}\NormalTok{ Value)) }\SpecialCharTok{+}
  \FunctionTok{geom\_histogram}\NormalTok{(}\AttributeTok{fill =} \StringTok{"dodgerblue"}\NormalTok{, }\AttributeTok{color =} \StringTok{"white"}\NormalTok{) }\SpecialCharTok{+}
  \FunctionTok{facet\_wrap}\NormalTok{(}\SpecialCharTok{\textasciitilde{}}\NormalTok{ Variable, }\AttributeTok{scales =} \StringTok{"free"}\NormalTok{) }\SpecialCharTok{+}
  \FunctionTok{labs}\NormalTok{(}\AttributeTok{title =} \StringTok{"Distribution of Numeric Variables"}\NormalTok{)}\SpecialCharTok{+}\FunctionTok{theme\_minimal}\NormalTok{()}

\CommentTok{\# Print the histograms}
\FunctionTok{print}\NormalTok{(histograms)}
\end{Highlighting}
\end{Shaded}

\begin{verbatim}
## `stat_bin()` using `bins = 30`. Pick better value with `binwidth`.
\end{verbatim}

\begin{verbatim}
## Warning: Removed 59 rows containing non-finite values (`stat_bin()`).
\end{verbatim}

\includegraphics{analysis_files/figure-latex/unnamed-chunk-8-1.pdf}

\begin{Shaded}
\begin{Highlighting}[]
\CommentTok{\# Visualize the distribution of factor variables}
\NormalTok{factor\_vars }\OtherTok{\textless{}{-}} \FunctionTok{names}\NormalTok{(data)[}\FunctionTok{sapply}\NormalTok{(data, is.factor)]}

\CommentTok{\# Create bar plots for factor variables}
\NormalTok{barplots }\OtherTok{\textless{}{-}}\NormalTok{ data }\SpecialCharTok{\%\textgreater{}\%}
  \FunctionTok{select}\NormalTok{(}\FunctionTok{all\_of}\NormalTok{(factor\_vars)) }\SpecialCharTok{\%\textgreater{}\%}
  \FunctionTok{pivot\_longer}\NormalTok{(}\FunctionTok{everything}\NormalTok{(), }\AttributeTok{names\_to =} \StringTok{"Variable"}\NormalTok{, }\AttributeTok{values\_to =} \StringTok{"Value"}\NormalTok{) }\SpecialCharTok{\%\textgreater{}\%}
  \FunctionTok{ggplot}\NormalTok{(}\FunctionTok{aes}\NormalTok{(}\AttributeTok{x =}\NormalTok{ Value)) }\SpecialCharTok{+}
  \FunctionTok{geom\_bar}\NormalTok{(}\AttributeTok{fill =} \StringTok{"dodgerblue"}\NormalTok{, }\AttributeTok{color =} \StringTok{"white"}\NormalTok{) }\SpecialCharTok{+}
  \FunctionTok{facet\_wrap}\NormalTok{(}\SpecialCharTok{\textasciitilde{}}\NormalTok{ Variable, }\AttributeTok{scales =} \StringTok{"free"}\NormalTok{) }\SpecialCharTok{+}
  \FunctionTok{labs}\NormalTok{(}\AttributeTok{title =} \StringTok{"Distribution of Factor Variables"}\NormalTok{)}\SpecialCharTok{+}\FunctionTok{theme\_minimal}\NormalTok{()}

\CommentTok{\# Print the bar plots}
\FunctionTok{print}\NormalTok{(barplots)}
\end{Highlighting}
\end{Shaded}

\includegraphics{analysis_files/figure-latex/unnamed-chunk-9-1.pdf}

\hypertarget{what-is-the-effect-of-age-at-heart-attack-on-the-survival-time-of-heart-attack-patients}{%
\subsubsection{What is the effect of age-at-heart-attack on the survival
time of heart attack
patients?}\label{what-is-the-effect-of-age-at-heart-attack-on-the-survival-time-of-heart-attack-patients}}

\begin{Shaded}
\begin{Highlighting}[]
\CommentTok{\# Load necessary libraries for survival analysis}
\FunctionTok{library}\NormalTok{(survival)}
\FunctionTok{library}\NormalTok{(mice)}
\end{Highlighting}
\end{Shaded}

\begin{verbatim}
## 
## Attaching package: 'mice'
\end{verbatim}

\begin{verbatim}
## The following object is masked from 'package:stats':
## 
##     filter
\end{verbatim}

\begin{verbatim}
## The following objects are masked from 'package:base':
## 
##     cbind, rbind
\end{verbatim}

\begin{Shaded}
\begin{Highlighting}[]
\FunctionTok{library}\NormalTok{(survminer)}
\end{Highlighting}
\end{Shaded}

\begin{verbatim}
## Loading required package: ggpubr
\end{verbatim}

\begin{verbatim}
## 
## Attaching package: 'survminer'
\end{verbatim}

\begin{verbatim}
## The following object is masked from 'package:survival':
## 
##     myeloma
\end{verbatim}

\begin{Shaded}
\begin{Highlighting}[]
\CommentTok{\# Create an imputation model}
\NormalTok{imputation\_model }\OtherTok{\textless{}{-}} \FunctionTok{mice}\NormalTok{(data, }\AttributeTok{method =} \StringTok{"pmm"}\NormalTok{, }\AttributeTok{m =} \DecValTok{5}\NormalTok{, }\AttributeTok{maxit =} \DecValTok{100}\NormalTok{, }\AttributeTok{seed =} \DecValTok{123}\NormalTok{)}
\end{Highlighting}
\end{Shaded}

\begin{verbatim}
## Warning: Number of logged events: 6800
\end{verbatim}

\begin{Shaded}
\begin{Highlighting}[]
\CommentTok{\# Impute the missing values}
\NormalTok{imputed\_data }\OtherTok{\textless{}{-}} \FunctionTok{complete}\NormalTok{(imputation\_model)}
\end{Highlighting}
\end{Shaded}

\begin{Shaded}
\begin{Highlighting}[]
\NormalTok{heart\_data }\OtherTok{\textless{}{-}}\NormalTok{ imputed\_data }

\CommentTok{\# Convert still\_alive variable to integer}
\NormalTok{heart\_data}\SpecialCharTok{$}\NormalTok{still\_alive }\OtherTok{\textless{}{-}} \FunctionTok{as.integer}\NormalTok{(}\FunctionTok{as.character}\NormalTok{(heart\_data}\SpecialCharTok{$}\NormalTok{still\_alive))}

\CommentTok{\# Perform survival analysis using the Cox proportional hazards model}
\NormalTok{surv\_model }\OtherTok{\textless{}{-}} \FunctionTok{coxph}\NormalTok{(}\FunctionTok{Surv}\NormalTok{(survival, still\_alive) }\SpecialCharTok{\textasciitilde{}}\NormalTok{age\_at\_heart\_attack, }\AttributeTok{data =}\NormalTok{ heart\_data)}

\CommentTok{\# Summarize the results of the survival analysis}
\FunctionTok{summary}\NormalTok{(surv\_model)}
\end{Highlighting}
\end{Shaded}

\begin{verbatim}
## Call:
## coxph(formula = Surv(survival, still_alive) ~ age_at_heart_attack, 
##     data = heart_data)
## 
##   n= 133, number of events= 44 
## 
##                        coef exp(coef) se(coef)    z Pr(>|z|)    
## age_at_heart_attack 0.06365   1.06572  0.01840 3.46 0.000541 ***
## ---
## Signif. codes:  0 '***' 0.001 '**' 0.01 '*' 0.05 '.' 0.1 ' ' 1
## 
##                     exp(coef) exp(-coef) lower .95 upper .95
## age_at_heart_attack     1.066     0.9383     1.028     1.105
## 
## Concordance= 0.65  (se = 0.042 )
## Likelihood ratio test= 11.58  on 1 df,   p=7e-04
## Wald test            = 11.97  on 1 df,   p=5e-04
## Score (logrank) test = 11.88  on 1 df,   p=6e-04
\end{verbatim}

This output shows the results of a Cox proportional hazards model, which
is used to model the relationship between survival time and one or more
predictor variables. In this case, the predictor variable is
\texttt{age\_at\_heart\_attack}. The model was fit using data from a
dataset called \texttt{heart\_data}, with 133 observations and 44
events.

The coefficient for \texttt{age\_at\_heart\_attack} is 0.05931, which
means that for each one-unit increase in
\texttt{age\_at\_heart\_attack}, the hazard ratio (i.e., the
instantaneous risk of the event occurring) increases by a factor of
exp(0.05931) = 1.06110. In other words, as age at heart attack
increases, the risk of still being alive (as indicated by the
\texttt{still\_alive} variable) also increases.

The p-value for the \texttt{age\_at\_heart\_attack} coefficient is
0.00103, which is statistically significant at the 0.05 level. This
suggests that there is a significant relationship between age at heart
attack and survival time.

The concordance value of 0.627 indicates that the model has moderate
predictive accuracy.

Overall, this model suggests that age at heart attack is a significant
predictor of survival time in this dataset.

\begin{Shaded}
\begin{Highlighting}[]
\CommentTok{\# Visualize the survival curves based on age groups}
\FunctionTok{ggsurvplot}\NormalTok{(}\FunctionTok{survfit}\NormalTok{(surv\_model), }\AttributeTok{data =}\NormalTok{ heart\_data, }\AttributeTok{pval =} \ConstantTok{TRUE}\NormalTok{,}
           \AttributeTok{conf.int =} \ConstantTok{TRUE}\NormalTok{,}
  \AttributeTok{surv.median.line =} \StringTok{"hv"}\NormalTok{,}
  \AttributeTok{ggtheme =} \FunctionTok{theme\_light}\NormalTok{(),}
  \AttributeTok{palette =} \FunctionTok{c}\NormalTok{(}\StringTok{"\#E7B800"}\NormalTok{, }\StringTok{"\#2E9FDF"}\NormalTok{),}
  \AttributeTok{xlim =} \FunctionTok{c}\NormalTok{(}\DecValTok{0}\NormalTok{, }\DecValTok{60}\NormalTok{),}
  \AttributeTok{xlab =} \StringTok{"Time in months"}\NormalTok{,}
  \AttributeTok{ylab =} \StringTok{"Survival probability"}\NormalTok{,}
  \AttributeTok{title =} \StringTok{"Kaplan{-}Meier Estimate of Survival"}\NormalTok{,}
  \AttributeTok{subtitle =} \StringTok{"Heart Attack Data"}\NormalTok{)}
\end{Highlighting}
\end{Shaded}

\begin{verbatim}
## Warning in .pvalue(fit, data = data, method = method, pval = pval, pval.coord = pval.coord, : There are no survival curves to be compared. 
##  This is a null model.
\end{verbatim}

\begin{verbatim}
## Warning in .add_surv_median(p, fit, type = surv.median.line, fun = fun, :
## Median survival not reached.
\end{verbatim}

\includegraphics{analysis_files/figure-latex/unnamed-chunk-12-1.pdf}

Here is an example of how you can stratify the analysis into standard vs
experimental groups, display the strata using the \texttt{summary}
function, plot the strata using \texttt{ggsurvplot}, and perform a
log-rank test to compare the survival curves of the two groups:

\begin{Shaded}
\begin{Highlighting}[]
\CommentTok{\# Load the necessary libraries}
\FunctionTok{library}\NormalTok{(tidyverse)}
\FunctionTok{library}\NormalTok{(survival)}
\FunctionTok{library}\NormalTok{(survminer)}

\CommentTok{\# Create a Surv object to represent the survival time and censoring information}
\NormalTok{survival\_object }\OtherTok{\textless{}{-}} \FunctionTok{with}\NormalTok{(heart\_data, }\FunctionTok{Surv}\NormalTok{(survival, still\_alive))}

\CommentTok{\# Fit a Kaplan{-}Meier model stratified by group using the survfit function}
\NormalTok{fit\_stratified }\OtherTok{\textless{}{-}} \FunctionTok{survfit}\NormalTok{(survival\_object }\SpecialCharTok{\textasciitilde{}}\NormalTok{ alive\_at\_1, }\AttributeTok{data =}\NormalTok{ heart\_data)}

\CommentTok{\# Display the strata using the summary function}
\FunctionTok{summary}\NormalTok{(fit\_stratified)}
\end{Highlighting}
\end{Shaded}

\begin{verbatim}
## Call: survfit(formula = survival_object ~ alive_at_1, data = heart_data)
## 
##                 alive_at_1=0 
##   time n.risk n.event survival std.err lower 95% CI upper 95% CI
##   0.50     99       1    0.990  0.0100        0.970        1.000
##   0.75     98       1    0.980  0.0141        0.952        1.000
##   1.00     97       1    0.970  0.0172        0.937        1.000
##   3.00     96       1    0.960  0.0198        0.922        0.999
##  15.00     85       1    0.948  0.0225        0.905        0.994
##  19.00     78       1    0.936  0.0253        0.888        0.987
##  19.50     74       1    0.923  0.0280        0.870        0.980
##  20.00     73       1    0.911  0.0303        0.853        0.972
##  21.00     71       1    0.898  0.0325        0.837        0.964
##  22.00     69       1    0.885  0.0345        0.820        0.955
##  40.00     19       1    0.838  0.0559        0.736        0.955
## 
##                 alive_at_1=1 
##   time n.risk n.event survival std.err lower 95% CI upper 95% CI
##   0.03     33       1   0.9697  0.0298       0.9129        1.000
##   0.25     32       4   0.8485  0.0624       0.7346        0.980
##   0.50     28       5   0.6970  0.0800       0.5566        0.873
##   0.75     23       5   0.5455  0.0867       0.3995        0.745
##   1.00     18       5   0.3939  0.0851       0.2580        0.601
##   1.25     13       1   0.3636  0.0837       0.2316        0.571
##   2.00     12       2   0.3030  0.0800       0.1806        0.508
##   3.00     10       1   0.2727  0.0775       0.1562        0.476
##   4.00      9       1   0.2424  0.0746       0.1326        0.443
##   5.00      8       3   0.1515  0.0624       0.0676        0.340
##   7.00      5       1   0.1212  0.0568       0.0484        0.304
##   7.50      4       1   0.0909  0.0500       0.0309        0.267
##  10.00      3       1   0.0606  0.0415       0.0158        0.232
##  28.00      2       1   0.0303  0.0298       0.0044        0.209
##  34.00      1       1   0.0000     NaN           NA           NA
## 
##                 alive_at_1=2 
##      time n.risk n.event survival std.err lower 95% CI upper 95% CI
\end{verbatim}

\begin{Shaded}
\begin{Highlighting}[]
\CommentTok{\# Generate a Kaplan{-}Meier curve for each stratum using the ggsurvplot function from the survminer package}
\FunctionTok{ggsurvplot}\NormalTok{(fit\_stratified, }\AttributeTok{data =}\NormalTok{ heart\_data)}
\end{Highlighting}
\end{Shaded}

\includegraphics{analysis_files/figure-latex/unnamed-chunk-14-1.pdf}

\begin{Shaded}
\begin{Highlighting}[]
\CommentTok{\# Perform a log{-}rank test to compare the survival curves of the two groups}
\FunctionTok{survdiff}\NormalTok{(survival\_object }\SpecialCharTok{\textasciitilde{}}\NormalTok{ alive\_at\_1, }\AttributeTok{data =}\NormalTok{ heart\_data)}
\end{Highlighting}
\end{Shaded}

\begin{verbatim}
## Call:
## survdiff(formula = survival_object ~ alive_at_1, data = heart_data)
## 
##               N Observed Expected (O-E)^2/E (O-E)^2/V
## alive_at_1=0 99       11   37.907    19.099   150.148
## alive_at_1=1 33       33    5.781   128.158   160.702
## alive_at_1=2  1        0    0.312     0.312     0.322
## 
##  Chisq= 161  on 2 degrees of freedom, p= <2e-16
\end{verbatim}

The first table shows the number of observations (\texttt{N}), the
number of observed events (\texttt{Observed}), the expected number of
events under the null hypothesis (\texttt{Expected}), and two test
statistics (\texttt{(O-E)\^{}2/E} and \texttt{(O-E)\^{}2/V}) for each
group defined by the \texttt{alive\_at\_1} variable.

The second line shows the overall chi-squared test statistic
(\texttt{Chisq}) with its degrees of freedom (\texttt{df}) and p-value
(\texttt{p}). In this case, the p-value is very small (less than 2e-16),
indicating that there is a statistically significant difference between
the survival curves of the groups defined by \texttt{alive\_at\_1}.

In summary, these results suggest that there is a significant difference
in survival between the groups defined by the \texttt{alive\_at\_1}
variable.

\end{document}
